\section{Hagan Rowlenstino/1174040}
\subsection{Background}
Python di desain sebagai bahasa pemrograman yang dapat digunakan sehari-hari. Pencipta python ,Guido van Rossum, telah menulis seri lengkap tentang sejarah bahasa tersebut.Python diciptakan di awal 1990 di CWI \'(the Centrum voor Wiskunde and Informatica), tempat kelahiran ALGOL \'(Algorithmic Language 68 ). Sebelumnya, Rossum juga telah mengerjakan bahasa pemrograman ABC, yang dikembangkan di  CWI sebagai bahasa pengajaran yang menekankan kejelasan. Walaupun project ABC telah di tutup , Rossum banyak belajar dari hal tersebut saat dia mulai membuat Python sebagai alat untuk multimedia dan project penelitian sistem operasi. Dia ingin Python mempunyai tingkatan yang cukup tinggi agar mudah untuk dibaca dan ditulis, juga mirip dengan Java, dan menawarkan portabilitas serta error model yang terdefinisi dengan baik.
\linebreak
\linebreak
Python juga kaya akan vocabulary yang berguna untuk membuat algoritma yang kompleks dengan efisien dikarenakan punya dictionaries yang memiliki string yang kuat dan assosiasi array yang fleksibel. Python menggabungkan antara fleksibilitas tingkat tinggi, kemampuan membaca, dan interface yang terdefinisi dengan baik. Kombinasi tersebut membuat Python cocok untuk menyelesaikan masalah komputasi non-algoritma seperti integrase dengan web, format data, ataw hardware kelas rendah. Python mudah untuk dipelajari karena strukturnya sederhana dan sintaksnya jelas, punya library yang portable dan dapat digunakan di beda perangkat,dan dapat terintegrasi dengan bahasa pemrograman lain seperti C, C++, dan Java.
\subsection{Problems}
\begin{enumerate}
\item Banyak pemrograman yang penggunaannya kompleks
\end{enumerate}
\subsection{Objective and Contribution}
\subsubsection{Objective}
\begin{enumerate}
\item Dapat memudahkan pemrograman dengan bahasa pemrograman yang tepat
\end{enumerate}
\subsubsection{Contribution}
\begin{enumerate}
\item Menggunakan Python sebagai bahasa pemrograman
\end{enumerate}
\subsection{Scoop and Environment}
\begin{enumerate}
\item Mengimplementasikan Python dalam pemrograman
\end{enumerate}
