\section{Hagan Rowlenstino/1174040}
\subsection{Teori}
\begin{enumerate}
	\item tipe data teks : ada string yaitu kumpulan karakter dan char adalah karakter. penulisannya harus diapit dengan tanda petik 1,2, ataupun 3
   ('..'), (".."), ('''...'''), ("""...""")

   tipe data angka : ada float yaitu bilangan pecahan dan integer yaitu bilangan bulat. penulisannya yaitu dengan menginisialisasikan nama
   variable lalu masukkan angka (x = 30)

   tipe data boolean : tipe yang memiliki dua nilai yaitu true dan false. penggunaannya huruf pertamanya harus kapital True dan False.

   \item input().inisialisasikan input tersebut x = input() lalu print(x)

   \item +,*,-,/. misal a = '10' maka integerr = int(a) dan misal a= 10 maka stringg = string(a)

   \item while : untuk perulangan yang tidak pasti

   \begin{verbatim}

  i = 0
	while True:
    if i < 10:
        print "Saat ini i bernilai: ", i
        i = i + 1
    elif i >= 10:
        break
   
   for : untuk perulangan yang pasti
	for i in range(0, 10):
    print i
    \end{verbatim}
    \item 
    \begin{verbatim}
    if kondisi:
	hasil

   dan
   if kondisi:
	hasil
	if kondisi:
	    hasil
	\end{verbatim}
	\item type error = ubah tipe str jadi int

	\item taruh try : diatas sintaks yang ingin diketahui jika terjadi error lalu enter dan tulis except: lalu tenkan enter 
dan masukkan tulisaan yang akan ditampilkan.
	\begin{verbatim}
	a = 2
	b = 'as'
	try:
    	print(a + b)
	except TypeError:
    	print("Integer dan String Tidak Dapat
    	 Dijumlah Karena Berbeda Tipe")
	\end{verbatim}

\end{enumerate}
\subsection{Keterampilan Pemrograman}
\begin{enumerate}
	\item \lstinputlisting{src/chapter2/1174040_1.py}

	\item \lstinputlisting{src/chapter2/1174040_2.py}

	\item \lstinputlisting{src/chapter2/1174040_3.py}

	\item \lstinputlisting{src/chapter2/1174040_4.py}

	\item \lstinputlisting{src/chapter2/1174040_5.py}

	\item \lstinputlisting{src/chapter2/1174040_6.py}

	\item \lstinputlisting{src/chapter2/1174040_7.py}

	\item \lstinputlisting{src/chapter2/1174040_8.py}

	\item \lstinputlisting{src/chapter2/1174040_9.py}

	\item \lstinputlisting{src/chapter2/1174040_10.py}
	
	\item \lstinputlisting{src/chapter2/1174040_11.py}
\end{enumerate}
\subsection{Keterampilan Penanganan Error}
\begin{enumerate}
	\item TypeError yaitu error di dalam tipe data disaat melakukan substring dan ingin memasukkannya ke dalam kondisi for 
	yang hanya menerima tipe int. jadi harus merubah tipe inputan yaitu string menjadi integer.

	\item \lstinputlisting{src/chapter2/1174040_2err.py}
\end{enumerate}
